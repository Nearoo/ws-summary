\section{Wichtige Diskrete Verteilungen}

\subsection{Diskrete Gleichverteilung}
Die \textit{diskrete Gleichverteilung} existiert nur auf einer endlichen Menge. Sie gehört zu einer ZV $X$ mit Wertebereich $\mathcal{W}$ und Gewichtsfunktion 
$$p_X(x_k) = P[X=x_k]=\frac{1}{N} \mbox{   für } k=1,\dots, N$$

\subsection{Unabhängige 0-1 Experimente}
Wir betrachten eine Folge gleichartiger Experimente, die alle nur mit Erfolg oder Misserfolg enden können und betrachten die Ereignisse $A_i = \{\mbox{Erfolg beim }i\mbox{-ten Experiment}\}$. Wir nehmen an, dass alle $A_i$ unabhängig sind und dass $P[A_i]=p$ für alle $i$. Wir können nun eine Indikatorfunktion $Y_i = I_{A_i}$ für jedes $i$ definieren, und danach die Folge von Ereignissen als Folge von 0 und 1 codieren. Dies werden wir für die nächsten Verteilungen brauchen.